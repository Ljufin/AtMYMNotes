\documentclass[10pt,a4paper]{article}
\usepackage[utf8]{inputenc}
\usepackage{amsmath}
\usepackage{amsfonts}
\usepackage{amssymb}
\begin{document}

\begin{center}
\textbf{Calculus for Vector-Valued Functions}
\end{center}

\paragraph{Def:} A function $f: \mathbb{R}^n \to \mathbb{R}^m$ is  \textit{vector-valued}. Such functions have the form
$$ f(x_1, \dots, x_n) = 
\begin{pmatrix}
f_1(x_1, \dots, x_n)\\
\vdots\\
\vdots\\
f_m(x_1, \dots, x_n)
\end{pmatrix}$$ 

\paragraph{Def:} Let $a = (a_1, \dots, a_n)$ and let $b = (b_1, \dots, b_n)$ be two points in $\mathbb{R}^n$. Then the \textit{distance} between $a$ and $b$, denoted by $|a-b|$, is
$$ |a-b| = \sqrt{(a_1-b_1)^2+\dots+(a_n-b_n)^2}$$

\paragraph{Def:} The \textit{length} of $a$ is defined by
$$ |a| = \sqrt{a_1^2+\dots+a_n^2}$$

\paragraph{Note:} The length of $a$ is the same as the distance between $a$ and the origin.

\paragraph{Note:} You may have noticed that there is a slight difference between vectors and points. A point is a location in space while a vector can be treated as a set of instructions to get from the origin to its associated point. More generally, remember that vectors in $\mathbb{R}^n$ are a particular case of the mathematical object called a vector. These vectors form a vector space over $\mathbb{R}$ while points do not. To illustrate this, consider the whether it makes any sense to add two points together or to scale a point. All these operations are meaningful in the context of vector spaces, but are meaningless when applied to locations.

\end{document}