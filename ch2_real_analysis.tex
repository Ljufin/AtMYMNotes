\documentclass[10pt,a4paper]{article}
\usepackage[utf8]{inputenc}
\usepackage{amsmath}
\usepackage{amsfonts}
\usepackage{amssymb}
\begin{document}

\begin{center}
\textbf{Real Analysis}
\end{center}

\paragraph{Summary:} The fundamental concepts in calculus are limits, continuity, differentiation, and integration. In turn, all these ideas are built off $\epsilon$ and $\delta$ based reasoning. A good foundation in real analysis will cover up to the uniform convergence of functions.
\begin{itemize}
\item \textbf{Basic Object:} The Real numbers.
\item \textbf{Basic Maps:} Continuous and Differentiable functions.
\item \textbf{Basic Goal:} The Fundamental Theorem of Calculus.
\end{itemize}

\paragraph{Def:} A function $f: \mathbb{R} \to \mathbb{R}$ has a \textit{limit} $L$ at the point $a$ if given $\epsilon >0$ there is a $\delta > 0$ such that $\forall x$ with $0 < |x-a| < \delta$ we have $|f(x)-L| < \epsilon$. This is denoted by 
$$ \lim_{x \to a}f(x) = L$$

\paragraph{Note:} The definition of the limit says that in order for $f(x)$ to be close to $L$, $x$ must be close to $a$. We can take the required restriction for $\epsilon$, $|f(x)-L| < \epsilon$, and find what our $\delta$ must be starting from $0 < |x-a| < \delta$.

\paragraph{Def:} A function $f: \mathbb{R} \to \mathbb{R}$ is \textit{continuous} at $a$ if 
$$ \lim_{x \to a} f(x) = f(a)$$

\paragraph{Note:} In order for the limit to exist, the right and left limits must be the same value.

\paragraph{Def:} A function $f:\mathbb{R} \to \mathbb{R}$ is continuous at $a$ if given any $\epsilon>0$, there is some $\delta>0$ such that $\forall x$ with $0 < |x-a| < \delta$, we have $|f(x) - f(a)| < \epsilon$. 

\paragraph{Def:} A function $f: \mathbb{R} \to \mathbb{R}$ is differentiable at $a$ if 
$$ \lim_{x \to a} \frac{f(x)-f(a)}{x-a}$$
exists. This limit is called the \textit{derivative} and is denoted by $f'(a)$ or $\frac{df}{dx}(a)$.

\paragraph{Note:} The slope of a tangent line to a function $f$ at $a$ is given by 
$$ f'(a) = \lim_{x \to a} \frac{f(x)-f(a)}{x-a}$$

\paragraph{Note:} Functions with sharp points such as $f(x) = |x|$ do not have derivatives at the sharp points.

\paragraph{Def:} Let $f(x)$ be a real-valued function defined on the closed interval $[a,b]$. For each positive integer $n$, let the lower sum of $f(x)$ be
$$ L(f,n) = \sum_{k=1}^n f(l_k) \Delta t$$
and the upper sum be
$$ U(f,n) = \sum_{k=1}^n f(u_k)\Delta t$$

\paragraph{Note:} the lower sum is the sum of the areas of the rectangles below our curve while the upper sum is the sum of the areas of the rectangles below our curve.

\paragraph{Def:} A real-valued function $f(x)$ with the domain $[a,b]$ is said to be integrable if the following condition holds:
$$ \lim_{n \to \infty} L(f,n) = \lim_{n \to \infty} U(f,n)$$
If these limits exist and are equal, we denote the quantity(both limits are the same real number) by 
$$ \int_a^b f(x)dx$$
and call it the \textit{integral} of $f(x)$.

\paragraph{Fundamental Theorem of Calculus} Let $f(x)$ be a real-valued continuous function defined on the closed interval $[a,b]$ and define
$$ F(x) = \int_a^x f(t)dt$$
Then:
\begin{enumerate}
\item The function $F(x)$ is differentiable and 
$$ \frac{dF(x)}{dx} = \frac{d\int_a^xf(t)dt}{dx} = f(x)$$
\item If $G(x)$ is a real-valued differentiable function defined on the closed interval $[a,b]$ whose derivative is
$$ \frac{dG(x)}{dx} = f(x)$$
then
$$ \int_a^b f(x)dx = G(b) - G(a)$$
\end{enumerate}

\paragraph{Note:} This theorem essentially states that the derivative and the integral are opposites of each other.

\paragraph{Def:} Let $f_n: [a,b] \to \mathbb{R}$ be a sequence of functions
$$ f_1(x),f_2(x),f_3(x), \dots$$
defined on an interval $[a,b] = \{x | a \leq x \leq b\}$. This sequence, denoted $\{f_n(x)\}$ will \textit{converge pointwise} to a function $f(x): [a,b] \to \mathbb{R}$ if $\forall \alpha \in [a,b]$,
$$ \lim_{n \to \infty} f_n(\alpha) = f(\alpha)$$

\paragraph{Note:} In $\epsilon$ and $\delta$ notation, we would say that $\{f_n(x)\}$ converges pointwise to $f(x)$ if $\forall \alpha \in [a,b]$ and given any $\epsilon > 0$, there is a positive integer $N$ such that $\forall n \geq N$, we have $|f(\alpha) - f_n(\alpha)|<\epsilon$.

\paragraph{Note:} It is important to note that the pointwise limit of a sequence of continuous functions need not be continuous. What this means is that a bunch of continuous functions can actually converge to a discontinuous function.

\paragraph{Def:} A sequence of functions $f_n:[a,b] \to \mathbb{R}$ will converge \textit{uniformly} to a function $f:[a,b] \to \mathbb{R}$ if given any $\epsilon >0$, $\exists N > 0$ such that $\forall n \geq N$, we have
$$ |f(x)-f_n(x)| < \epsilon$$
for all points $x$.

\paragraph{Note:} Uniform convergence is stronger than pointwise convergence since the sequence will have to uniformly converge for $N$ for all points $x$. A common way to phrase uniform convergence is to say that the sequence converges independently of $x$.

\paragraph{Thm.} Let

\end{document}