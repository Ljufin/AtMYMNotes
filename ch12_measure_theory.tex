\documentclass[10pt,a4paper]{article}
\usepackage[utf8]{inputenc}
\usepackage{amsmath}
\usepackage{amsfonts}
\usepackage{amssymb}
\begin{document}

\begin{center}
\textbf{Lebesgue Integration}
\end{center}

\paragraph{Def:} For the interval $E = [a,b]$, the \textit{length} of $E$ is 
$$ \ell(E) = b-a$$

\paragraph{Note:} Finding the length of intervals is easy, but we can also find the length of sets that are not intervals based on what we know about the easy lengths.

\paragraph{Def:} Let $E$ be any subset of $\mathbb{R}$. A countable collection of intervals $\{ I_n \}$, with each $I_n = [a_n, b_n]$, \textit{covers} the set $E$ if
$$ E \subset \bigcup I_n$$

\paragraph{Def:} For any set $E$ in $\mathbb{R}$, the \textit{outer measure} of $E$ is 
$$ m^*(E) = \inf\left \{ \sum (b_n-a_n)\right \}$$
such that the collection of intervals $\{ [a_n, b_n]\}$ covers E.

\paragraph{Note:} This definition is basically saying that the outer measure is the smallest covering of the set.

\paragraph{Def:} A set $E$ is measurable if for every set $A$,
$$ m^*(A) = m^*(A \cap E) + m^*(A-E)$$
The \textit{measure} of a measurable set $E$, denoted by $m(E)$, is $m^*(E)$.

\paragraph{Def:} The \textit{inner measure} of a set $E$ is
$$ m_*(E) = \sup \left \{ \sum (b_n-a_n) | E \supset \bigcup I_n \text{ and } I_n = [a_n,b_n] \text{ with } a_n \leq b_n \right \}$$

\paragraph{Note:} The intuition for the inner measure is that we are filling the interval in question with a collection of closed intervals.

\paragraph{Thm.} If $m^*(E) < \infty$, then the set $E$ is measurable if and only if
$$ m^*(E) = m_*(E)$$

\paragraph{Example:} The Cantor set is constructed bu taking the unit interval and dividing it into thirds and leaving out the middle third. This process is repeated recursively forever. The result is a set that has measure zero since the length of the intervals goes to zero. The Cantor set is also uncountable because every element can be expressed as an infinite sequence of 0's and 2's.

\paragraph{Def:} For a countable collection of disjoint measurable sets $A_i$, the function
$$ \sum a_i \chi_{A_i}$$
is called a \textit{step function}.

\paragraph{Def:} Let $E$ be a measurable set and let $\sum  a_i \chi_{A_i}$ be a step function. Then
$$ \int_E \left( \sum  a_i \chi_{A_i} \right)= \sum a_i m(A_i \cap E)$$

\paragraph{Def:} A function $f: E \to \mathbb{R} \cup \infty \cup -\infty$ is measurable if its domain $E$ is measurable and if, for any fixed $\alpha \in \mathbb{R} \cup \infty \cup -\infty$, $\{x \in E | f(x) = \alpha \}$ is measurable.

\paragraph{Def:} Let $f$ be a measurable function on $E$. Then the \textit{Lebesgue integral} of $f$ on $E$ is
$$ \int_E f = \inf \left\{  \int_E \sum a_i\chi_{A_i} | \forall x \in E, \sum a_i \chi_{A_i} \geq f(x) \right\}$$

\paragraph{Note:} The intuition for the Lebesgue integral is to approximate the function using a step function and then sum the product of the function values and the measures at each step.

\paragraph{Lebesgue Dominating Convergence Theorem:}  Let $g(x)$ be a Lebesgue integrable function on a measurable set $E$ and let $\{f_n(x)\}$ be a sequence of Lebesgue integrable functions $E$ with $|f_k(x)| \leq g(x)$ for all $x \in E$ and such that there is a pointwise limit of the $f_k(x)$, i.e. there is a function $f(x)$ with
$$ f(x) = \lim_{k \to \infty} f_k(x)$$
Then 
$$ \int_E \lim_{k \to \infty} f_k(x) = \lim_{k \to \infty} \int_E f_k(x)$$

\paragraph{Exercises:}
\paragraph{1.} Let $X$ be any countable set of real numbers. Show that $m(X)=0$.
\begin{flushleft}
\textit{Answer:}\\
Let $X = \{x_i \}_{i=1}^n$ be a countable set of real numbers. For every $x_i$, we can cover the real number with the interval $(x_i-2^{-i}\epsilon, x_i+2^{-i}\epsilon) = A_i$ for $\epsilon > 0$. Therefore, we have constructed a new set such that
$$ X \subseteq \bigcup_{i=1}^n A_i$$
Since the measure of $X$ must be smaller or equal to the measure of the new set, we have
$$ m(X) \leq m\left(\bigcup_{i=1}^n A_i\right) \leq \sum_{i=1}^n \ell(A_i)$$
Substituting in the length of the intervals,
$$ \sum_{i=1}^n \ell(A_i) = \sum_{i=1}^n \left[(x_i+2^{-i}\epsilon)-(x_i-2^{-i}\epsilon)\right] = \sum_{i=1}^n 2^{1-i}\epsilon = \epsilon + \sum_{i=1}^n 2^{-i}\epsilon = 2\epsilon$$
This gives us
$$ m(X) \leq 2\epsilon$$
which means that $m(X) =0$. $\blacksquare$
\end{flushleft}

\paragraph{2.} Let $f(x)$ and $g(x)$ be two Lebesgue integrable functions, both with domain $E$. Suppose that the set
$$ A = \{x \in E | f(x) \neq g(x) \}$$
has measure zero. What can be said about $\int_Ef(x)$ and $\int_Eg(x)$?
\begin{flushleft}
\textit{Answer:}\\
By the definition of a Lebesgue integrable function, $E$ must be measurable, $\{x \in E|f(x) = \alpha \}$ and $\{x \in E|g(x) = \beta\}$ are measurable for fixed $\alpha$ and $\beta$. $A$ having measure zero means that $f(x)$ and $g(x)$ are very similar except at arbitrarily small intervals. Therefore, $\int_Ef(x)=\int_Eg(x)$.
\end{flushleft}



\end{document}